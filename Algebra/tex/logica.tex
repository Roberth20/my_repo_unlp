\chapter{Logica}

\section{Escribir en el lenguaje simbólico – Tablas de verdad}

\textit{El propósito de los siguientes ejercicios es  familiarizarnos con la notación simbólica del cálculo proposicional y con el uso de esquemas y cuantificadores. Además, es importante que ejerciten el hacer tablas de verdad para poder establecer después el valor de verdad de proposiciones compuestas a partir del valor de las proposiciones atómicas que las integran.}\\

\textbf{Ejercicio 1.} Determinar si los siguientes enunciados son proposiciones. Escribir en lenguaje simbólico aquellos que sean
proposición e indicar su valor de verdad. Y si algún enunciado no es proposición, explicar porqué.

\begin{enumerate}
	\item 8 es par y 6 es impar. 

	Son proposiciones, llamando a las proposiciones como $p = \text{ 8 es par}$ y $q = \text{ 6 es impar}$. Construimos la tabla de verdad para la preposici\'on. \textbf{Recordamos} $\wedge$ es la \textbf{conjunci\'on} o \textit{"y}". 
	
	\begin{displaymath}
		\begin{array}{|c c|c|}
			p & q & p \land q\\
			\hline 
			V & V & V\\
			V & F & F\\
			F & V & F\\
			F & F & F\\
		\end{array}
	\end{displaymath}

	Encontramos que la proposici\'on es inequívocamente falsa, debido a los valores de $p$ y $q$.
	
	\item 8 es par o 6 es impar.
	
	Es una proposici\'on. Llamando $p = \text{ 8 es par}$ y $q = \text{ 6 es impar}$, a su vez \textbf{recordando} a la \textbf{disyunci\'on} u "o" como $\vee$, tenemos:
	
	\begin{displaymath}
		\begin{array}{|c c|c|}
			p & q & p \vee q\\
			\hline 
			V & V & V\\
			V & F & V\\
			F & V & V\\
			F & F & F\\
		\end{array}
	\end{displaymath}

	As\'i, encontramos que la proposici\'on es verdadera y se cumple cuando $p$ es verdadera o $q$ es falsa.
	
	\item 4 es par y 2 no divide a 5.
	
	Es proposici\'on. Llamando $p = \text{ 4 es par}$ y $q = \text{ 2 no divide a 5}$. Se trata de una conjunci\'on, por lo tanto, 	
	
	\begin{displaymath}
		\begin{array}{|c c|c|}
			p & q & p \land q\\
			\hline 
			V & V & V\\
			V & F & F\\
			F & V & F\\
			F & F & F\\
		\end{array}
	\end{displaymath}

	Y la proposici\'on ser\'a verdadera siempre y cuando ambas afirmaciones sean correctas.
	
	\item $x < 2$. 
	
	No es una proposici\'on, esto se debe a que no es posible determinar el valor de verdad de la expresi\'on sin antes fijar un valor a la variable $x$.
	
	\item Si 8 es par y 6 impar, o bien 4 es par o 2 divide a 8.
	
	Es una proposici\'on. Nombramos a las proposiciones como:
	
	\begin{align*}
		p &= \text{ 8 es par} & q &= \text{ 6 es impar}\\
		r &= \text{ 4 es par} & s &= \text{ 2 divide a 8}
	\end{align*}

	La proposici\'on tendr\'a la forma: $(p \wedge q) \rightarrow (r \vee s)$ y la tabla de verdad es:
	
	\begin{displaymath}
		\begin{array}{|c|c|c|c|c|c|c|}
			p & q & r & s & p \wedge q & r \vee s & (p \wedge q) \rightarrow (r \vee s)\\
			\hline 
			V & V & V & V & V & V & V\\
			V & V & V & F & V & V & V\\
			V & V & F & V & V & V & V\\
			V & V & F & F & V & F &	F\\
				
			V & F & V & V & F & V & V\\
			V & F & V & F & F & V & V\\
			V & F & F & V & F & V & V\\
			V & F & F & F & F & F &	V\\	
			
			F & V & V & V & F & V & V\\
			F & V & V & F & F & V & V\\
			F & V & F & V & F & V & V\\
			F & V & F & F & F & F &	V\\
			
			F & F & V & V & F & V & V\\
			F & F & V & F & F & V & V\\
			F & F & F & V & F & V & V\\
			F & F & F & F & F & F &	V\\	
		\end{array}
	\end{displaymath}

	Para la misma se utilizo la definici\'on de tabla de verdad del \textbf{condicional} $\rightarrow$ que tiene la forma:

	\begin{displaymath}
		\begin{array}{|c c|c|}
			p & q & p \rightarrow q\\
			\hline 
			V & V & V\\
			V & F & F\\
			F & V & V\\
			F & F & V\\
		\end{array}
	\end{displaymath}

	Y por lo tanto, concluimos que la proposici\'on ser\'a siempre verdadera cuando $q$ sea falsa.
	
	\item Hace frío.
	
	No es una proposici\'on, se trata de una opini\'on subjetiva, implicando que no puede cuantificarse su valor de verdad.
	
	\item 10 es múltiplo de 5 pero no de 3. 
	
	Es una proposici\'on. Llamando las proposiciones $p = \text{ 10 es m\'ultiplo de 5}$ y $\neg q = \text{ 10 es multiplo de 3}$. Construimos la tabla de verdad
	
	\begin{displaymath}
		\begin{array}{|c c|c|}
			p & \neg q & p \wedge \neg q\\
			\hline 
			V & F & F\\
			V & V & V\\
			F & F & F\\
			F & V & F\\
		\end{array}
	\end{displaymath}

	Encontramos que es verdad siempre que $p$ sea verdad y $q$ falso.
\end{enumerate}

\textbf{Ejercicio 2}. Dadas la siguientes proposiciones, reescribirlas utilizando “necesario” y “suficiente”.

\begin{enumerate}
	\item Si un número es múltiplo de 3 entonces su cuadrado es múltiplo de 9. 
	
	Reescribiendo como condicional y recordando que, la estructura es: p es \textit{suficiente} para q, o, q es \textit{necesaria} para p. As\'i, llamando $p =$ un n\'umero es m\'ultiplo de 3 y $q = $ su cuadrado es m\'ultiplo de 9. Las dos formas son:
	\begin{itemize}
		\item Un n\'umero m\'ultiplo de 3 es \textit{suficiente} para que su cuadrado sea m\'ultiplo de 9.
		\item Si el cuadrado de un n\'umero es m\'ultiplo de 9 es \textit{necesario} que el n\'umero sea m\'ultiplo de 3
	\end{itemize}

	\item Un número es múltiplo de 4 sólo si es divisible por 2.
	
	Llamando $p =$ n\'umero es m\'ultiplo de 4 y $q =$ divisible por 2. Entonces,
	\begin{itemize}
		\item Si un n\'umero es m\'ultiplo de 4 es \textit{suficiente} para que sea divisible por 2.
		\item Si un n\'umero es divisible por 2 es \textit{necesario} que el m\'ultiplo del n\'umero sea 4
	\end{itemize}

	\item Un número es múltiplo de 7 si es múltiplo de 21.
	
	Llamando $q =$ n\'umero es m\'ultiplo de 7 y $p =$ n\'umero m\'ultiplo de 21, utilizando la estructura detallada previamente, las oraciones son
	\begin{itemize}
		\item Si un n\'umero es m\'ultiplo de 21 es \textit{suficiente} para que sea m\'ultiplo de 7
		\item Un n\'umero es m\'ultiplo de 7 es necesario que sea m\'ultiplo de 21.
	\end{itemize}

	\item Enunciar el recíproco, el contrarrecíproco y el contrario del enunciado en b).
	
	Recordando que el \textbf{rec\'iproco} es $q \rightarrow p$ si $p \rightarrow q$, el \textbf{contrarrec\'iproco} es $\neg q \rightarrow \neg p$ y el \textbf{contrario} es $\neg p \rightarrow \neg q$. De esta manera, tenemos que las expresiones son:
	\begin{itemize}
		\item \textbf{Rec\'iproco:} Un n\'umero es divisible por 2 si es m\'ultiplo de 4.
		\item \textbf{Contrarrec\'iproco:} Si un n\'umero no es divisible por 2 entonces no es m\'ultiplo de 4.
		\item \textbf{Contrario:} Un n\'umero no es m\'ultiplo de 4 si no es divisible por 2.
	\end{itemize}
\end{enumerate}

\textbf{Ejercicio 3.} Construir las tablas de verdad de las siguientes fórmulas y clasificarlas en tautologías ,contradicciones y
contingencias.

\begin{enumerate}
	\item $\neg p \rightarrow (q \vee \neg p)$
	
	La tabla de verdad correspondiente es
	
	\begin{displaymath}
		\begin{array}{|c c|c|c|}
			\neg p & q & q \vee \neg p & \neg p \rightarrow (q \vee \neg p)\\
			\hline 
			F & V & V & F\\
			F & F & F & V\\
			V & V & V & V\\
			V & F & V & V\\
		\end{array}
	\end{displaymath}
	
	Debido a que su veracidad depende los valores de las proposiciones, concluimos que es una \textbf{contingencia.}
	
	\item $((p \wedge q) \rightarrow p) \rightarrow q$
	
	Repitiendo el procedimiento previo,
	
	\begin{displaymath}
		\begin{array}{|c c|c|c|c|}
			p & q & q \wedge p & (p \wedge q) \rightarrow p & ((p \wedge q) \rightarrow p) \rightarrow q \\
			\hline 
			V & V & V & V & V\\
			V & F & F & V & F\\
			F & V & F & V & F\\
			F & F & F & V & F\\
		\end{array}
	\end{displaymath}

	Se trata de una contingencia.
	
	\item $(\neg p \rightarrow q) \rightarrow (\neg q \rightarrow p)$
	
	Construimos la tabla de verdad
	
	\begin{displaymath}
		\hspace*{-1cm}
		\begin{array}{|c c|c|c|c|c|c|}
			p & q & \neg p & \neg q & \neg p \rightarrow q & \neg q \rightarrow p & (\neg p \rightarrow q) \rightarrow (\neg q \rightarrow p) \\
			\hline 
			V & V & F & F & V & V & V\\
			V & F & F & V & V & V & V\\
			F & V & V & F & V & V & V\\
			F & F & V & V & F & F & V\\
		\end{array}
	\end{displaymath}
	
	Debido que en todos los casos siempre se llega al mismo resultado de verdad y que el mismo es Verdad, concluimos que se trata de una tautolog\'ia.
	
	\item $((p \wedge q) \vee (r \wedge \neg q)) \leftrightarrow ((\neg p \wedge \neg q) \vee (\neg r \wedge \neg q))$
	
	Llamando $P = ((p \wedge q) \vee (r \wedge \neg q)) \leftrightarrow ((\neg p \wedge \neg q) \vee (\neg r \wedge \neg q))$, construimos la tabla.
	
	\begin{displaymath}
		\hspace*{-1cm}
		\begin{array}{|c c c|c|c|c|c|c|c|c|}
			p & q & r & p \wedge q & r \wedge \neg q & \neg p \wedge \neg q & \neg r \wedge \neg q & (p \wedge q) \vee (r \wedge \neg q) & (\neg p \wedge \neg q) \vee (\neg r \wedge \neg q) & P \\
			\hline 
			V & V & V & V & F & F & F & V & F & F\\
			V & V & F & V & F & F & F & V & F & F\\
			V & F & V & F & V & F & F & V & F & F\\
			V & F & F & F & F & F & V & F & V & F\\
			F & V & V & F & F & F & F & F & F & V\\
			F & V & F & F & F & F & F & F & F & V\\
			F & F & V & F & V & V & F & V & V & V\\
			F & F & F & F & F & V & V & F & V & F\\
		\end{array}
	\end{displaymath}

	Encontramos que se trata de una contingencia.
	
\end{enumerate}

\section{Esquemas l\'ogicos - Cuantificadores}

\textbf{Ejercicio 4.} Simbolizar utilizando universo, esquemas, cuantificadores y conectivos lógicos:

\begin{enumerate}
	\item Todos los números son enteros.
	
	Definiendo un universo con todos los n\'umeros enteros $\mathbb{N}$, entonces:
	\begin{equation*}
		(\forall n)(P(n))\qquad : P(n) = n \text{ pertence a } \mathbb{Z}
	\end{equation*}

	\item Existen números impares o no todos los números son pares.
	
	Definiendo los n\'umeros $n \in \mathbb{Z}$ y $m \in \mathbb{Z}$. Llamando a las proposiciones $p = n \text{ impar}$ y $q = m \text{ par}$, construimos la proposici\'on $$p \vee \neg q$$ De esta manera, la expresi\'on simb\'olica es $$(\exists n)(\forall m)(p \vee \neg q)$$
	
	\item Para todo par de números reales, si su producto es uno entonces uno es el inverso del otro.
	
	Definiendo un par de n\'umeros $(x,y) \in \mathbb{R}$, y las proposiciones
	\begin{align*}
		p(x,y) &\equiv x \cdot y = 1 & q(x,y) &\equiv x = \frac{1}{y} 
	\end{align*}
	Podemos escribir la expresi\'on como $$(\forall x)(\forall y)(p(x,y) \rightarrow q(x,y))$$
	
	\item Dado dos números reales, existe un número real mayor a la suma de ambos.
	
	Para el universo de los n\'umeros reales, definimos tres valores $(x,y,z) \in \mathbb{R}$, la cuantificaci\'on de l proposici\'on ser\'a $$p(x,y,z) \equiv z > x+y$$ As\'i, la expresi\'on l\'ogica es $$(\forall x)(\forall y)(\exists z)(p(x,y,z))$$
\end{enumerate}

\textbf{Ejercicio 5.} Escribir en el lenguaje corriente las siguientes proposiciones, siendo el universo el conjunto de números
reales y los esquemas definidos como sigue:
\begin{align*}
	p(x)&: x es par & q(x)&: x es divisible por 2 & r(x)&: x> 0\\
	p(x,y)&: y > x & q(x,y)&: x + y = 0
\end{align*}

\begin{enumerate}
	\item $(\forall x)(p(x) \rightarrow q(x))$
	
	\textit{Para todo n\'umero real, si es par entonces el mismo es divisible por 2}
	
	\item $(\exists y)(\forall x)p(x,y)$
	
	\textit{Existe un n\'umero real mayor a cualquier otro.}
	
	\item $(\forall x)(\exists y)p(y, x+3)$
	
	\textit{Para todo n\'umero real sumado 3, existe otro que es mayor.}
	
	\item $(\forall x)\left(r(x) \rightarrow (\exists y)(\neg r(y) \wedge q(x,y))\right)$
	
	\textit{Para todo n\'umero real positivo, existe un n\'umero negativo para el cu\'al su suma es igual a cero.}
\end{enumerate}

\section{Implicaciones y leyes de equivalencia}

\textit{Es importante incorporar y aprender a manejar las equivalencias e implicaciones lógicas que veremos a continuación. Son de utilidad para simplificar y operar de forma más efectiva con proposiciones y esquemas proposicionales más complejos. Como los enunciados en matemática son de la forma "si pasa tal cosa"(hipótesis), ”entonces pasa tal otra"(conclusión), es fundamental que incorporen las equivalencias lógicas necesarias para la negación de una implicación o condicional $p \rightarrow q$. Además, como muchas veces nuestras propiedades son enunciados cuantificados, es igual de
fundamental que incorporen las equivalencias lógicas que se usan para negar cuantificadores.}

Las siguientes son algunas implicaciones l\'ogicas:

\begin{enumerate}
	\item \textbf{Doble negaci\'on:} $p \Leftrightarrow \neg (\neg p)$
	\item \textbf{Leyes conmutativas:} $p \wedge q \Leftrightarrow q \wedge p\qquad$ y $\qquad p \vee q \Leftrightarrow q \vee p$
	\item \textbf{Leyes asociativas:} $(p \vee q) \vee r \Leftrightarrow p \vee (q \vee r)\qquad$ y $\qquad (p \wedge q) \wedge r \Leftrightarrow p \wedge (q \wedge r)$
	\item \textbf{Leyes distributivas:} $(p \vee q) \wedge r \Leftrightarrow (p \wedge r) \vee (q \wedge r)\quad$ y $\quad (p \wedge q) \vee r \Leftrightarrow (p \vee r) \wedge (q \vee r)$
	\item \textbf{Leyes de Morgan:} $\neg(p \vee q) \Leftrightarrow \neg p \wedge \neg q \qquad$ y $\qquad \neg(p \wedge q) \Leftrightarrow \neg p \vee \neg q$
	\item \textbf{Ley de implicaci\'on:} $p \rightarrow q \Leftrightarrow \neg p \vee q$
\end{enumerate}

\textbf{Ejercicio 6.}
\begin{enumerate}
	\item Probar las equivalencias de 5, mostrando que las proposiciones que se obtienen reemplazando $\Leftrightarrow$ por $\leftrightarrow$ son
	tautologías.
	
	Expresando la primera con la forma del bicondicional, $\neg(p \vee q) \leftrightarrow \neg p \wedge \neg q$ procedemos a elaborar la tabla de verdad.
	
	\begin{displaymath}
		\hspace*{-1cm}
		\begin{array}{|c c|c|c|c|}
			p & q & p \vee q & \neg p \wedge \neg q & \neg(p \vee q) \leftrightarrow \neg p \wedge \neg q\\
			\hline 
			V & V & V & F & V\\
			V & F & V & F & V\\
			F & V & V & F & V\\
			F & F & F & V & V\\
		\end{array}
	\end{displaymath}
	
	Como se trata de una tautolog\'ia, confirmamos que la primera equivalencia es correcta. Repetimos para la segunda, cambiando la equivalencia por un bicondicional, tenemos la expresi\'on $\neg(p \wedge q) \leftrightarrow \neg p \vee \neg q$, y su tabla de verdad es:
	
	\begin{displaymath}
		\hspace*{-1cm}
		\begin{array}{|c c|c|c|c|}
			p & q & p \wedge q & \neg p \vee \neg q & \neg(p \wedge q) \leftrightarrow \neg p \vee \neg q\\
			\hline 
			V & V & V & F & V\\
			V & F & F & V & V\\
			F & V & F & V & V\\
			F & F & F & V & V\\
		\end{array}
	\end{displaymath}
	
	Al igual que el resultado anterior, se trata de una tautolog\'ia que comprueba la equivalencia.
	
	\item Deducir la equivalencia de la implicación con el contrarrecíproco: $p \rightarrow q \Leftrightarrow \neg q \rightarrow \neg p$
	
	Construimos la tabla de verdad para la expresi\'on, reemplazando $\Leftrightarrow$ por $\leftrightarrow$.
	
	\begin{displaymath}
		\hspace*{-1cm}
		\begin{array}{|c c|c|c|c|}
			p & q & p \leftrightarrow q & \neg q \leftrightarrow \neg p & p \rightarrow q \leftrightarrow \neg q \rightarrow \neg p\\
			\hline 
			V & V & V & V & V\\
			V & F & F & F & V\\
			F & V & V & V & V\\
			F & F & V & V & V\\
		\end{array}
	\end{displaymath}

	Como se trata de una tautolog\'ia, se demuestra la equivalencia entre ambas
	
\end{enumerate}

Las siguientes son algunas implicaciones lógicas, que incorporaremos como herramientas para hacer pruebas más
complejas:

\begin{enumerate}
	\item \textbf{Adici\'on:} $p \Rightarrow (p \vee q)$
	\item \textbf{Simplificaci\'on:} $(p \wedge q) \Rightarrow p$
	\item \textbf{Modus Ponens: } $p \wedge (p \rightarrow q) \Rightarrow q$
	\item \textbf{Modus Tolens: } $(p \vee q) \wedge \neg p \Rightarrow q$
	\item \textbf{Silogismo hipot\'etico: } $(p \rightarrow q) \wedge (q \rightarrow r) \Rightarrow p \rightarrow r$
\end{enumerate}

\textbf{Ejercicio 7.} Probar las implicaciones 3 y 5, mostrando que las proposiciones que se obtienen reemplazando $\Rightarrow$ por $\rightarrow$
son tautologías.

Haciendo el reemplazo en Modus ponens, la tabla de verdad es

\begin{displaymath}
	\hspace*{-1cm}
	\begin{array}{|c c|c|c|c|}
		p & q & p \rightarrow q & p \wedge (p \rightarrow q) & p \wedge (p \rightarrow q) \rightarrow q\\
		\hline 
		V & V & V & V & V\\
		V & F & F & F & V\\
		F & V & V & F & V\\
		F & F & V & F & V\\
	\end{array}
\end{displaymath}

Comprobando as\'i la implicaci\'on. Y para el silogismo hipot\'etico

\begin{displaymath}
	\hspace*{-0cm}
	\begin{array}{|c c c|c|c|c|c|c|}
		p & q & r & p \rightarrow q & q \rightarrow r & (p \rightarrow q) \wedge (q \rightarrow r) & p \rightarrow r & (p \rightarrow q) \wedge (q \rightarrow r) \rightarrow p \rightarrow r\\
		\hline 
		V & V & V & V & V & V & V & V\\
		V & V & F & V & F & F & F & V\\
		V & F & V & F & V & F & V & V\\
		V & F & F & F & V & F & F & V\\
		F & V & V & V & V & V & V & V\\
		F & V & F & V & F & F & V & V\\
		F & F & V & V & V & V & V & V\\
		F & F & F & V & V & V & V & V\\
	\end{array}
\end{displaymath}

\section{Negaci\'on de cuantificadores y condicionales}

Teniendo en cuenta las equivalencias:

\begin{align}
	\neg (\forall x)p(x) &\Leftrightarrow (\exists x)(\neg p(x)) & &\text{y} & \neg(\exists x)p(x) &\Leftrightarrow (\forall x)(\neg p(x))
\end{align}

\textbf{Ejercicio 8.} Encontrar expresiones equivalentes para la negación de los esquemas del ejercicio 5.

\begin{enumerate}
	\item Planteando la negaci\'on y utilizando la equivalencia
	\begin{equation*}
		\neg(\forall x)(p(x) \rightarrow q(x)) \Leftrightarrow (\exists x)(\neg(p(x) \rightarrow q(x)))
	\end{equation*}
	Aplicando la Ley de Implicaci\'on $$(\exists x)(\neg(p(x) \rightarrow q(x))) \Leftrightarrow (\exists x)(\neg(\neg p(x) \vee q(x)))$$ 
	Luego, la Ley de Morgan $$(\exists x)(\neg(\neg p(x) \vee q(x))) \Leftrightarrow (\exists x)(\neg (\neg p(x)) \wedge \neg q(x))$$ 
	Y por doble negaci\'on, $$(\exists x)(\neg (\neg p(x)) \wedge \neg q(x)) \Leftrightarrow (\exists x)(p(x) \wedge \neg q(x))$$ 
	As\'i, se lee como \textit{Existe un n\'umero par que no es divisible por 2}
	
	\item Planteando la equivalencia de negaci\'on $$\neg (\exists y)(\forall x)p(x,y) \Leftrightarrow (\forall y)(\exists x)(\neg p(x,y))$$ Y se lee, \textit{Para todo n\'umero, existe otro que es menor.}
	
	\item Procediendo como en los incisos anteriores, $$\neg (\forall x)(\exists y) p(y, x+3) \Leftrightarrow (\exists x)(\forall y)(\neg p(y, x+3))$$ Y se lee, \textit{Para todo n\'umero existe otro que es 3 unidades menor}
	
	\item Finalmente, $$\neg (\forall x)(r(x) \rightarrow (\exists y)(\neg r(y) \wedge q(x,y))) \Leftrightarrow (\exists x)\neg(r(x) \rightarrow (\exists y)(\neg r(y) \wedge q(x,y)))$$
	Utilizando la Ley de Implicaci\'on $$(\exists x)\neg(r(x) \rightarrow (\exists y)(\neg r(y) \wedge q(x,y))) \Leftrightarrow (\exists x)\neg(\neg r(x) \vee (\exists y)(\neg r(y) \wedge q(x,y)))$$
	Luego, por Ley de Morgan $$(\exists x)\neg(\neg r(x) \vee (\exists y)(\neg r(y) \wedge q(x,y))) \Leftrightarrow (\exists x)(\neg (\neg r(x)) \wedge \neg (\exists y)(\neg r(y) \wedge q(x,y)))$$
	Empleando la doble negaci\'on y la negaci\'on de cuantificadores $$(\exists x)(\neg (\neg r(x)) \wedge \neg (\exists y)(\neg r(y) \wedge q(x,y))) \Leftrightarrow (\exists x)(r(x) \wedge (\forall y)\neg(\neg r(y) \wedge q(x,y)))$$
	Aplicando la Ley de Morgan de nuevo $$(\exists x)(r(x) \wedge (\forall y)\neg(\neg r(y) \wedge q(x,y))) \Leftrightarrow (\exists x)(r(x) \wedge (\forall y)(\neg (\neg r(y)) \vee \neg q(x,y)))$$
	Y por doble negaci\'on $$(\exists x)(r(x) \wedge (\forall y)(\neg (\neg r(y)) \vee \neg q(x,y))) \Leftrightarrow (\exists x)(r(x) \wedge (\forall y)(r(y) \vee \neg q(x,y)))$$ 
	Y se lee, \textit{Existe un n\'umero positivo que al sumar con cualquier otro n\'umero, el segundo es positivo o la suma es diferente de cero.}
\end{enumerate}

\textbf{Ejercicio 9.} A partir de la leyes de equivalencia, probar la siguiente equivalencia para la negación de una implicación: 
\begin{equation}
	\neg (p \rightarrow q) \Leftrightarrow p \wedge \neg q
\end{equation}

Desde la implicaci\'on, partimos aplicando la Ley de la implicaci\'on $$\neg (p \rightarrow q) \Leftrightarrow \neg (\neg p \vee q)$$
Luego, utilizamos la Ley de Morgan $$\neg (\neg p \vee q) \Leftrightarrow \neg(\neg p) \wedge \neg q$$ 
Y por doble negaci\'on $$\neg(\neg p) \wedge \neg q \Leftrightarrow p \wedge \neg q$$

\textbf{Ejercicio 10.} Definir universo y esquemas para simbolizar la siguiente proposición:\\

\textit{Para todo par de números reales, si su suma es 16 y su producto es 9, entonces uno de ellos es 5}\\

Negar la proposición anterior en forma simbólica y traducirla al lenguaje coloquial.

Definiendo el universo $U = \mathbb{R}$, siendo $(x,y) \in U$, definimos las proposiciones 
\begin{align*}
	p(x,y) &: x + y = 16 & q(x,y) &: x\cdot y = 9 & r(x) &: x = 5
\end{align*}

Construimos el esquema l\'ogico $$\forall (x,y)((p(x,y) \wedge q(x,y)) \rightarrow (r(x) \vee r(y)))$$

La negaci\'on ser\'a $$\neg \forall (x,y)((p(x,y) \wedge q(x,y)) \rightarrow (r(x) \vee r(y))) \Leftrightarrow \exists (x,y)\neg((p(x,y) \wedge q(x,y)) \rightarrow (r(x) \vee r(y)))$$ 
Aplicando la negaci\'on de la implicaci\'on (1.2), $$\exists (x,y)\neg((p(x,y) \wedge q(x,y)) \rightarrow (r(x) \vee r(y))) \Leftrightarrow \exists (x,y)((p(x,y) \wedge q(x,y)) \wedge \neg (r(x) \vee r(y)))$$
Y por Ley de Morgan $$\exists (x,y)((p(x,y) \wedge q(x,y)) \wedge \neg (r(x) \vee r(y))) \Leftrightarrow \exists (x,y)((p(x,y) \wedge q(x,y)) \wedge (\neg r(x) \wedge \neg r(y)))$$
Leyendose como \textit{Existe un par de n\'umeros reales para los cuales su suma es 16, su multiplicaci\'on es 9 y ninguno es igual a 5}