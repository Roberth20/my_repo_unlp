\chapter{Teor\'ia de Conjuntos}

Los aspectos básicos de la Teoría de Conjuntos que estudiaremos en la materia y trabajaremos en esta guía son útiles porque nos permiten establecer un marco riguroso para desarrollar el estudio de los conjuntos numéricos que conocemos, y otras estructuras algebraicas de una manera formal.

A lo largo de estos ejercicios iremos aplicando lo visto en lógica: todo lo que hemos trabajado en relación a las reglas
de inferencia, a las equivalencias lógicas, a los razonamientos válidos, etc, será ahora una herramienta indispensable para
demostrar teoremas y propiedades de conjuntos.

Recordemos brevemente las notaciones que utilizamos para algunos conjuntos numéricos y que aparecen en esta
práctica:

\begin{itemize}
	\item El conjunto de \textbf{números reales}, que notamos por $\mathbb{R}$
	
	\item El conjunto de \textbf{números enteros}, que notamos por $\mathbb{Z}$
	
	\item El conjunto de \textbf{números naturales}, que notamos por $\mathbb{N}$.
\end{itemize}

\hspace*{-1.5cm}
\framebox[1.15\linewidth][c]{Utilizando la notación de conjuntos, tenemos que se dan las inclusiones siguientes: $\mathbb{N} \subset \mathbb{Z} \subset \mathbb{R}$}

\section{Definici\'on por extensi\'on y comprensi\'on - Conjunto Vac\'io}

\textit{En esta primer tanda de ejercicios trabajamos con el concepto de conjunto y su definición: tanto por extensión -exhibiendo la lista de sus elementos- como por comprensión -en donde se especifica una (o varias) propiedades que debe cumplir un elemento para pertenecer al conjunto en cuestión-. Es importante notar que podemos definir de muchas formas a un mismo conjunto por comprensión (¡incluso al conjunto vacío!).}

\textbf{Ejercicio 1.} Definir los siguientes conjuntos por extensión:

\begin{enumerate}
	\item $A = \{n \in \mathbb{N}: n \text{ divide a } 6\}$
	
	Los n\'umeros pertenecientes a los naturales que son divisores del 6 son 1, 2, 3 y 6. Por lo tanto, el conjunto $A$ definido por extensi\'on es $$A = {1, 2, 3, 6}$$
	
	\item $B = \{ k \in \mathbb{Z}: -5 < k < 10 \}$
	
	Como $k$ pertenece a los enteros, $B$ por extensi\'on es $$B = \{ -4, -3, -2, -1, 0, 1, 2, 3, 4, 5, 6, 7, 8, 9 \}$$
\end{enumerate}