\chapter{Teor\'ia de Conjuntos}

Los aspectos básicos de la Teoría de Conjuntos que estudiaremos en la materia y trabajaremos en esta guía son útiles porque nos permiten establecer un marco riguroso para desarrollar el estudio de los conjuntos numéricos que conocemos, y otras estructuras algebraicas de una manera formal.

A lo largo de estos ejercicios iremos aplicando lo visto en lógica: todo lo que hemos trabajado en relación a las reglas
de inferencia, a las equivalencias lógicas, a los razonamientos válidos, etc, será ahora una herramienta indispensable para
demostrar teoremas y propiedades de conjuntos.

Recordemos brevemente las notaciones que utilizamos para algunos conjuntos numéricos y que aparecen en esta
práctica:

\begin{itemize}
	\item El conjunto de \textbf{números reales}, que notamos por $\mathbb{R}$
	
	\item El conjunto de \textbf{números enteros}, que notamos por $\mathbb{Z}$
	
	\item El conjunto de \textbf{números naturales}, que notamos por $\mathbb{N}$.
\end{itemize}

\hspace*{-1.5cm}
\framebox[1.15\linewidth][c]{Utilizando la notación de conjuntos, tenemos que se dan las inclusiones siguientes: $\mathbb{N} \subset \mathbb{Z} \subset \mathbb{R}$}

\section{Definici\'on por extensi\'on y comprensi\'on - Conjunto Vac\'io}

\textit{En esta primer tanda de ejercicios trabajamos con el concepto de conjunto y su definición: tanto por extensión -exhibiendo la lista de sus elementos- como por comprensión -en donde se especifica una (o varias) propiedades que debe cumplir un elemento para pertenecer al conjunto en cuestión-. Es importante notar que podemos definir de muchas formas a un mismo conjunto por comprensión (¡incluso al conjunto vacío!).}\\

\textbf{Ejercicio 1.} Definir los siguientes conjuntos por extensión:

\begin{enumerate}
	\item $A = \{n \in \mathbb{N}: n \text{ divide a } 6\}$
	
	Los n\'umeros pertenecientes a los naturales que son divisores del 6 son 1, 2, 3 y 6. Por lo tanto, el conjunto $A$ definido por extensi\'on es $$A = {1, 2, 3, 6}$$
	
	\item $B = \{ k \in \mathbb{Z}: -5 < k < 10 \}$
	
	Como $k$ pertenece a los enteros, $B$ por extensi\'on es $$B = \{ -4, -3, -2, -1, 0, 1, 2, 3, 4, 5, 6, 7, 8, 9 \}$$
\end{enumerate}

\textbf{Ejercicio 2.} Definir por comprensión:

\begin{enumerate}
	\item El conjunto de los números naturales \textbf{impares}.
	
	Los impares son los n\'umeros que no son divisibles por 2/. De esta manera, $$C = \{ n \in \mathbb{N}, n \% 2 \neq 0\}$$ Donde la operaci\'on $\%$ es el modulo, o resto de $n$
	
	\item El conjunto $C = {4, 9, 16, 25, 36}$
	
	Observando el conjunto podemos notar que son n\'umeros al cuadrado, por lo tanto $$ C = \{ n^2 \in \mathbf{N} : 2 \leq n \leq 6 \}$$
\end{enumerate}

\textbf{Ejercicio 3.} Determinar si $A = B$ en los siguientes casos:

\begin{enumerate}
	\item $A = \{ z \in \mathbb{Z} : z \text{ es impar} \}$ y $B = \{ z \in \mathbb{Z}: \text{ existe } k \in \mathbb{Z} \text{ tal que } z = 2k + 1 \}$
	
	$A$ se define como todos los n\'umeros impares en $\mathbb{Z}$, por otro lado, $B$ es un conjunto definido por una operaci\'on que convierte cualquier n\'umero en impar. Como ambos son subconjuntos de $\mathbb{Z}$, y desarrollando por expresi\'on, ambos contienen los mismos elementos, entonces $A = B$
	
	\item $A = \emptyset$ y $B = \{\emptyset\}$
	
	$\emptyset$ representa el conjunto vac\'io, $A$ es igual a dicho conjunto, mientras que $B$ es un conjunto que tiene como elemento al conjunto vac\'io. Esta diferencia en definici\'on, nos permite concluir que $A \neq B$
	
	\item $A = \emptyset$ y $B = \{ x \in \mathbb{R}: x^2 + 1= 0 \}$
	
	Resolviendo la ecuaci\'on que define al conjunto $B$, encontramos que no existe n\'umero real que sea soluci\'on, siendo as\'i $B = \emptyset$. Por lo tanto, $A = B$
	
	\item $A = \{ y \in \mathbb{N}: 0 < y < 1 \}$ y $B = \{ y \in \mathbb{R}: 0 < y < 1 \}$
	
	Ambos conjuntos difieren del conjunto de n\'umeros del cual se definen, siendo as\'i que seg\'un cada definici\'on $A = \emptyset$ y $B$ contiene infinitos elementos. Por estas razones, concluimos que $A \neq B$
	
	\item $A = \{ -2, 0, 1, -1, 2, 3 \}$ y $B = \{ x \in \mathbb{R}: x \text{ es un n\'umero entero y } -\frac{5}{2} \leq x < \sqrt{9} \}$
	
	Como los dos son conjuntos finitos, podemos desarrollar $B$, as\'i $$B = \{-2, -1, 0, 1, 2\}$$ Ambos poseen los mismos elementos a excepci\'on del 3. Por lo tanto, $A \neq B$
\end{enumerate}

\section{Inclusión de conjuntos - Subconjuntos}

\textit{La relación inclusión nos sirve para “comparar” dos conjuntos; se trata de analizar si todos los elementos de uno 	son también elementos del otro. Es muy importante aprender a distinguir este concepto del de pertenencia y para eso
es que trabajaremos los siguientes ejercicios. Recuerden que hay que tener mucho cuidado en no confundir los símbolos
“$\in$” (“pertenece a”) y “$\subset$” (“está incluido en”).}\\

\textbf{Ejercicio 4.} Sea $A = \{1, 2, \{3\}, \{1, 2\}, -1\}$, decir si son verdaderas o falsas las siguientes relaciones. Justifique.

\begin{enumerate}
	\item $3 \in A$. 
	
	\textbf{Falso.} $\in$ hace referencia a "\textit{elemento} : \textit{conjunto}". De esta forma, el elemento 3, no se encuentra en $A$.
	
	\item $\{1, 2\}$ es subconjunto de $A$.
	
	\textbf{Verdadero.} Todos los elementos del conjunto se encuentran dentro de $A$
	
	\item $\{1, 2\} \in A$.
	
	\textbf{Verdadero.} En este caso, se cuestiona si el conjunto, como elemento, se encuentra dentro de $A$
	
	\item $\{3\} \subset A$ 
	
	\textbf{Falso.} El s\'imbolo de subconjunto $\subset$ hace referencia a "\textit{subconjunto}:\textit{conjunto}". El subconjunto formado por el elemento 3 no pertenece a $A$
	
	\item $\{\{3\}\} \subset A$
	
	\textbf{Verdadero.} Forma correcta del inciso anterior.
	
	\item $\emptyset \in A$.
	
	\textbf{Falso.} El elemento que representa al conjunto vac\'io no esta definido en $A$
	
	\item $\{-1, 2\} \subset A$
	
	\textbf{Verdadero.} Todos los elementos del conjunto se encuentran dentro de $A$
	
	\item $\emptyset \subset A$
	
	\textbf{Verdadero.} Por definici\'on el conjunto vac\'io es un subconjunto de todo conjunto.
	
	\item $\{1, 2, -1\} \in A$
	
	\textbf{Falso.} Dicho elemento no se encuentra en $A$
\end{enumerate}

\textbf{Ejercicio 5.} Sean $A = \{x \in \mathbb{Z}: x \text{ es multiplo de } 3\}$, $B = \{x \in \mathbb{Z}: x \text{ es multiplo de } 7\}$ y $C = \{x \in \mathbb{Z}: x \text{ es multiplo de } 21\}$. Probar que:

\begin{enumerate}
	\item $C \subset A$ y $A \neq C$
	
	Debido a que $(\forall c \in C)(c \text{ es multiplo de 3})$, tenemos que $\forall c \in C \wedge c \in A$. Concluimos que $C \subset A$. Sin embargo, no ocurre lo contrario, implicando que $A \neq C$, y lo comprobamos con el n\'umero 9, este pertenece a $A$ pero no a $C$
	
	\item $C \subset B$ y $B \neq C$.
	
	Siguiendo el mismo an\'alisis comprobamos y demostramos ambas afirmaciones
\end{enumerate}

\textbf{Ejercicio 6.} Sea $E$ el conjunto cuyos elementos son el conjunto $\mathbb{N}$ de los n\'umeros naturales y el conjunto $\mathbb{P}$ de los n\'umeros naturales pares.

\begin{enumerate}
	\item \textquestiondown Es $\{2, 4, 6\}$ subconjunto de \textit{E}?
	
	La expresi\'on que define la pregunta es $\{2, 4, 6\} \subset E$. Debido a que el conjunto no se encuentra en \textit{E}, la pregunta es negativa.
	
	\item \textquestiondown $\{\mathbb{N}\} \subset E$?
	
	El conjunto formado por el conjunto de los naturales es un subconjunto de \textit{E}, por lo tanto es afirmativo.
	
	\item \textquestiondown $\mathbb{N} \subset E$?
	
	En este caso es negativo, los naturales no son un subconjunto de \textit{E}. Es conveniente escribir de forma simbolica \textit{E}. $E = \{\mathbb{N}, \mathbb{P}\}$
	
	\item \textquestiondown $\mathbb{P} \in E$?
	
	Correcto, el conjunto de los n\'umeros pares es un elemento de \textit{E}
	
	\item \textquestiondown $\emptyset \subset E$? 
	
	Correcto, el conjunto vac\'io es parte de todos los conjuntos.
\end{enumerate}

\textbf{Ejercicio 7.} Analizar la validez, en general, de la afirmación \textbf{Si B $\in$ A y C $\subset$ B, entonces C $\subset$ A}

Esta afirmaci\'on es \textbf{falsa.} Como vimos en el ejercicio anterior, si un conjunto \textit{pertenece} a otro, implica que todo el conjunto es un elemento de otro. Como \textbf{C} es subconjunto de \textbf{B} pero \textbf{B} es un elemento de \textbf{A}, entonces \textbf{C} no puede ser un conjunto de \textbf{A}. Solo es posible si B $\subset$ A

\section{Conjunto de partes}

\textit{l conjunto de partes es la “prueba de fuego” para detectar si estamos trabajando correctamente los conceptos de
inclusión y pertenencia. Debemos comprender que la idea es que \textbf{pertenecer a} $P(A)$ equivale a \textbf{estar incluído en} $A$.}

\textbf{Ejercicio 8.} Sea $A = \{a, b, c, d\}$
\begin{enumerate}
	\item Hallar $P(A)$.
	
	Como $P(A)$ se define por todos los posibles subconjuntos de \textit{A}, entonces 
	\begin{align*}
		P(A) = &\{\emptyset, \{a\}, \{b\}, \{c\}, \{d\}, \{a, b\}, \{a, c\}, \{a, d\}, \{b, c\}, \{b, d\}, \{c, d\},\\
		 &\{a, b, c\}, \{a, b, d\}, \{b, c, d\}, A\}
	\end{align*}
	
	\item Decidir cu\'ales de las siguientes expresiones son verdaderas:
	\begin{enumerate}
		\item $\{a\} \in A$. \textbf{Falso}
		\item $\{a\} \in P(A)$. \textbf{Verdadero}
		\item $\emptyset \in P(A)$. \textbf{Verdadero}
		\item $\emptyset \subset P(A)$. \textbf{Verdadero}
		\item $b \in P(A)$. \textbf{Falso}
		\item $\{a, \{b, c\}\} \in P(A)$. \textbf{Falso}
	\end{enumerate}
\end{enumerate}

\textbf{Ejercicio 9.}
\begin{enumerate}
	\item Sea $A = \{1, \{2\}, \{\emptyset\}, \{1, 2\}\}$, hallar $P(A)$
	
	\begin{align*}
		P(A) =& \{ \emptyset, \{1\}, \{\{2\}\}, \{\{\emptyset\}\}, \{\{1, 2\}\}, \{1, \{2\}\}, \{1, \{\emptyset\}\}, \{1, \{1, 2\}\}, \\
		&\{\{2\}, \{\emptyset\}\}, \{\{2\}, \{1, 2\}\}, \{1, \{2\}, \{\emptyset\}\}, \{1, \{2\}, \{1, 2\}\},\\
		&\{\{2\}, \{\emptyset\}, \{1, 2\}\}, A \} 
	\end{align*}

	\item Hallar: $P(\emptyset)$ y $P(P(\emptyset))$
	
	\begin{align*}
		P(\emptyset) &= \{\emptyset, \{\emptyset\}\}\\
		P(P(\emptyset)) &= \{\emptyset, \{\emptyset\}, \{\{\emptyset\}\}, P(\emptyset)\}
	\end{align*}
\end{enumerate}

\textbf{Ejercicio 10.} Construir tres conjuntos \textit{A, B} y \textit{C} que cumplan simult\'aneamente las siguientes condiciones:
\begin{multicols}{3}
\begin{enumerate}
	\item $2 \in A$
	\item $\{1, 2, 3\} \in P(A)$
	\item $\{\{2\}\} \in P(A)$
	\item $\emptyset \in B$
	\item $\{6, \{10\}\} \subset B$
	\item $A \in C$
	\item $B \subset C$
\end{enumerate}
\end{multicols}

De la primera condici\'on, sabemos que 2 pertenece a \textit{A}, $A = \{2\}$. Luego, de la segunda condici\'on, como el conjunto formado por 1, 2 y 3 pertenece al conjunto de partes de \textit{A}. Entonces, el mismo es un subconjunto de \textit{A}. As\'i, $$A = \{1, 2, 3\}$$ Por la tercera condici\'on, el conjunto formado por $\{2\}$ es un subconjunto de \textit{A}. Por lo tanto, $$A = \{1, 2, 3, \{2\}\}$$ El conjunto vac\'io es un elemento de \textit{B}, quedando el mismo como $$B = \{\emptyset\}$$ Luego, el conjunto formado por 6 y $\{10\}$ es un subconjunto de \textit{B}, es decir, todos sus elementos son elementos de \textit{B}. $$B = \{\emptyset, 6, \{10\}\}$$ \textit{A} es un elemento de \textit{C}. Por lo tanto, todo el conjunto \textit{A} esta dentro de \textit{C} como un \'unico elemento $$C = \{A\} = \{\{1, 2, 3, \{2\}\}\}$$ Finalmente, \textit{B} es un subconjunto de \textit{C}, siendo as\'i y resumiendo la construcci\'on de los conjuntos como:

\begin{align*}
	A &= \{1, 2, 3, \{2\}\} & B &= \{\emptyset, 6, \{10\}\} & C &= \{\{1, 2, 3, \{2\}\}, \emptyset, 6, \{10\} \}
\end{align*}

\section{Operaciones con conjuntos}

\subsection{Uni\'on e intersecci\'on}

\textit{La unión e intersección de conjuntos son dos operaciones binarias que nos permiten crear, a partir de un par 
de conjuntos, nuevos conjuntos. Es importante no confundirlas, tanto a las operaciones como a las notaciones que se
	emplean para indicarlas. También es importante relacionarlas con los conectivos disyunción y conjunción que vimos en
	lógica.}\\

\textbf{Ejercicio 11.} Hallar la uni\'on $A \cup B$ en los siguientes casos:

\begin{enumerate}
	\item $A = \{x \in \mathbb{Z}: -2 \leq x \leq 8\} \qquad B = \{x \in \mathbb{Z}: -5 \leq x \leq 3\}$
	
	La uni\'on representa que ambos conjuntos unen todos sus elementos, de esta manera $$A \cup B = \{x \in \mathbb{Z}: -5 \leq x \leq 8\}$$
	
	\item $A = \{x \in \mathbb{N}: 1 \leq x < 8\} \qquad B = \{x \in \mathbb{N}: 8 < x \leq 12\}$
	
	As\'i, la uni\'on es $$A \cup B = \{x \in \mathbb{N}: 1 \leq x \leq 12\}$$
\end{enumerate}

\textbf{Ejercicio 12.} Hallar la intersecci\'on $A \cap B$ en los siguientes casos:

\begin{enumerate}
	\item $A = \{x \in \mathbb{R}: 0 \leq x \leq 6\} \qquad B = \{y \in \mathbb{N}: 0 < y \leq 10\}$
	
	La intersecci\'on es el conjunto formado por elementos que est\'an tanto en \textit{A} como en \textit{B}, as\'i $$A \cap B = \{y \in \mathbb{N}: 0 < y \leq 6\}$$
	
	\item $A = \{x \in \mathbb{R}: -1 < x \leq 1/4\} \qquad B = \{ z \in \mathbb{R}: -1 < z < 0 \text{ \'o } 0 < z < 3 \}$
	
	Aplicando el concepto anterior $$ A\cap B = \{ x \in \mathbb{R}: -1 < x < 0 \text{ \'o } 0 < x \leq 1/4 \}$$
\end{enumerate}

\textbf{Ejercicio 13.} Tomando los conjuntos: $A = \{1, 3, \{4, 5, 6\}, \{7\}\}; \quad B = \{ 3, 4, 5, b, c \};$ $C = \{ 0, b, \{c\}, 2, 3, 4 \} \text{ y } D = \{ \{1\}, \{2\}, \{7\}, b, c \}$. Hallar:

\begin{enumerate}
	\item $(A \cup B) \cap C$
	
	Primero, realizamos la uni\'on de \textit{A} y \textit{B} $$A \cup B = \{ 1, 3, 4, 5, b, c, \{3, 4, 6\}, \{7\} \}$$ Y ahora, su intersecci\'on con \textit{C} $$(A \cup B) \cap C = \{b, 3, 4\}$$
	
	\item $(A \cap B)\cup C$
	
	Repetimos procedimiento para todos los incisos, $$A \cap B = \{ 3 \}$$ $$(A \cap B)\cup C = \{3, 0, b, \{c\}, 2, 4\}$$
	
	\item $D \cap C \cap B$
	
	La intersecci\'on es una operaci\'on asociativa, por lo tanto $$D \cap C = \{b\}$$ $$D \cap C \cap B = \{b\}$$
	
	\item $A \cup (C \cup B)$
	$$C \cup B = \{0, b, \{c\}, 2, 3, 4, 5, c\}$$ $$A \cup (C \cup B) = \{1, \{4, 5, 6\}, \{7\}, 0, b, \{c\}, 2, 3, 4, 5, c\}$$
	
	\item $(A \cap B)\cup(C \cap D)$
	$$A \cap B = \{3\}$$ $$C \cap D = \{b\}$$ $$(A \cap B)\cup(C \cap D) = \{3, b\}$$
\end{enumerate}

\subsection{Complemento -Diferencia y Diferencia simétrica}

\textit{Tomar complemento de un conjunto es la operación que nos queda para analizar. Es fundamental tener presente
el universo respecto al cual complementamos (y que siempre lo indiquemos en los ejercicios). Vamos a trabajar en
reconocer a las restantes operaciones (diferencia y diferencia simétrica) caracterizándolas a partir de las operaciones
que ya estudiamos. Por ejemplo, podemos interpretar $A - B$ como la intersección de A con el complemento de B.}\\

\textbf{Ejercicio 14.} Sean $A = \{1, 2, 3\}; \quad B = \{3, 4, 1, b\}; \quad C = \{0, b, 2, 3\}$. Hallar: \textit{De ser necesario, considerar para este ejercicio como universo al conjunto $U = A \cup B \cup C$} 

\begin{enumerate}
	\item $A - B$
	
	La diferencia entre dos conjuntos, por intuici\'on, estar\'ia representada por todos los elementos de \textit{A} que no se encuentran en \textit{B}. Por notaci\'on de conjuntos, esto ser\'ia
	\begin{equation}
		A - B = A \cap B^c
	\end{equation}
	Donde $B^c$ es el conjunto complemento de \textit{B}, es decir, todos los elementos del universo que no son parte de \textit{B}. De esta forma, nuestro universo es $$U = \{1, 2, 3, 4, b, 0\}$$ Y la resta tiene como resultado
	\begin{align*}
		A - B &= A\cap B^c & B^c &= \{2, 0\}\\
		&= \{2\}
	\end{align*}
\end{enumerate}